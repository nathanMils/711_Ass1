\documentclass[conference]{IEEEtran}
\IEEEoverridecommandlockouts
% The preceding line is only needed to identify funding in the first footnote. If that is unneeded, please comment it out.
%Template version as of 6/27/2024

\usepackage{cite}
\usepackage{amsmath,amssymb,amsfonts}
\usepackage{algorithmic}
\usepackage{graphicx}
\usepackage{textcomp}
\usepackage{xcolor}
\def\BibTeX{{\rm B\kern-.05em{\sc i\kern-.025em b}\kern-.08em
    T\kern-.1667em\lower.7ex\hbox{E}\kern-.125emX}}
\begin{document}

\title{COS 711 Assignment 2 Report\\
}

\author{\IEEEauthorblockN{Nathan Opperman}
\IEEEauthorblockA{ \\
u21553832@tuks.co.za}
}

\maketitle

\begin{abstract}
The abstract should briefly summarise the purpose and findings of the report.
\end{abstract}

\begin{IEEEkeywords}
This probably is not necessary
\end{IEEEkeywords}

\section{Introduction}
The introduction sets the stage for the remainder of your report. You usually have very general statements here. The introduction prepares the reader for what to expect from reading your report. In general, the introduction should either contain or be a summary of your ENTIRE report. Keep the introduction concise, try to limit it to 1 page maximum

\section{Background}
A very high level discussion on the problem domain and the algorithms and/or approaches that you have used. This section is typically where the “base cases” of concepts that appear throughout the remainder of your report are discussed. It is also an ideal place to refer a reader to other sources containing relevant information on the topic which is outside the scope of your assignment. Remember to discuss very generally. After reading this section the marker should be able to determine whether or not you understand the techniques that you are using. Try to limit this section to 1 page maximum. Make sure you reference the relevant sources when discussing the building blocks of your project.

\section{Experimental Set-Up}
In this section you discuss how you approached, implemented and solved your assignment. Mention the values considered for the hyperparameters, how many simulations you have run, etc. After reading this section (in addition to the background) the reader should be able to replicate your experiments to obtain similar results to those obtained by you. Be very specific in your discussions in this section.

\section{Research Results}
This is the section where you report your results obtained from running the experiments as discussed in the experimental set-up section. You have to give, at the very least, the averages and the standard deviations for all the experiments simulations. Graphing your results is advisable, and no conclusions regarding the superiority of one approach over another can be made without some form of statistical reasoning. Training and testing errors have to be reported. Thoroughly discuss the results that you have obtained, and reason about why you obtained the results that you have. Answer questions like “are these results to be expected?” and “why these results occurred?” and “would different circumstances lead to different results?”

\section{Conclusions}
Very general conclusions about the assignment that you have done. This section “answers” the questions and issues that you have raised and investigated. This section is, in general, a summary of what you have done, what the results were, and finally what you concluded from these results. This is the final section in your document, so be sure that all the issues raised up until now are answered here. This is also the perfect section to discuss what you have learnt in doing this assignment.


\begin{thebibliography}{00}
\bibitem{b1} G. Eason, B. Noble, and I. N. Sneddon, ``On certain integrals of Lipschitz-Hankel type involving products of Bessel functions,'' Phil. Trans. Roy. Soc. London, vol. A247, pp. 529--551, April 1955.
\bibitem{b2} J. Clerk Maxwell, A Treatise on Electricity and Magnetism, 3rd ed., vol. 2. Oxford: Clarendon, 1892, pp.68--73.
\bibitem{b3} I. S. Jacobs and C. P. Bean, ``Fine particles, thin films and exchange anisotropy,'' in Magnetism, vol. III, G. T. Rado and H. Suhl, Eds. New York: Academic, 1963, pp. 271--350.
\bibitem{b4} K. Elissa, ``Title of paper if known,'' unpublished.
\bibitem{b5} R. Nicole, ``Title of paper with only first word capitalized,'' J. Name Stand. Abbrev., in press.
\bibitem{b6} Y. Yorozu, M. Hirano, K. Oka, and Y. Tagawa, ``Electron spectroscopy studies on magneto-optical media and plastic substrate interface,'' IEEE Transl. J. Magn. Japan, vol. 2, pp. 740--741, August 1987 [Digests 9th Annual Conf. Magnetics Japan, p. 301, 1982].
\bibitem{b7} M. Young, The Technical Writer's Handbook. Mill Valley, CA: University Science, 1989.
\bibitem{b8} D. P. Kingma and M. Welling, ``Auto-encoding variational Bayes,'' 2013, arXiv:1312.6114. [Online]. Available: https://arxiv.org/abs/1312.6114
\bibitem{b9} S. Liu, ``Wi-Fi Energy Detection Testbed (12MTC),'' 2023, gitHub repository. [Online]. Available: https://github.com/liustone99/Wi-Fi-Energy-Detection-Testbed-12MTC
\bibitem{b10} ``Treatment episode data set: discharges (TEDS-D): concatenated, 2006 to 2009.'' U.S. Department of Health and Human Services, Substance Abuse and Mental Health Services Administration, Office of Applied Studies, August, 2013, DOI:10.3886/ICPSR30122.v2
\bibitem{b11} K. Eves and J. Valasek, ``Adaptive control for singularly perturbed systems examples,'' Code Ocean, Aug. 2023. [Online]. Available: https://codeocean.com/capsule/4989235/tree
\end{thebibliography}

\end{document}
